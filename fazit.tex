\subsection{Zusammenfassung}
Ziel der Arbeit war es, ein generisches Routing-Protokoll für mobile Ad-Hoc-Netzwerke zu entwerfen. Weiterhin sollte die dafür entwickelte Lösung mit Hilfe einer Android-Anwendung auf Praktikabilität getestet werden. 
\\Nach einer kurzen Einleitung wurden die wichtigsten Grundlagen im Bereich des Shark-Frameworks und Routings erläutert. Für die Einordnung dieser Arbeit wurden anschließend wissenschaftliche Veröffentlichungen mit ähnlichen Problemstellungen vorgestellt. Darauf folgte die Vorstellung des grundlegenden Entwurfs. Dieser enthält die Merkmale eines semantischen Broadcast-Routings und die dafür benötigten theoretischen Vorüberlegungen, Anwendungsoberflächen und Datenbankmodelle. Weiterhin wurden der Aufbau des Protokolls und dessen Nachrichtenpakete präsentiert und aus verschiedenen Lösungsansätzen wurden die geeignetsten ausgewählt. Im Kapitel Entwurf wurde außerdem die Wechselwirkung zwischen dem Shark-Framework und der mobilen Anwendung SharkNet skizziert. 
\\Den zweiten Teil der Arbeit machte das Kapitel Implementierung aus. Da die mobile Anwendung komponentenbasiert entwickelt worden ist, enthält dieses Kapitel Komponentenbeschreibungen. In jeder Komponentenbeschreibung wurden die Aufgabe, die Struktur, die wichtigsten Codeauszüge, die Wiederverwendbarkeit und Tests der Komponente vorgestellt. 
\subsection{Fazit}
Der semantische Broadcast zwischen mehreren Geräten konnte erfolgreich realisiert werden. Damit konnte die technische Umsetzbarkeit des im Kapitel Entwurf beschriebenen Broadcast-Routings bewiesen werden. Hierbei ist jedoch einschränkend zu erwähnen, dass die Kommunikation zwischen Peers gerade bei hoher Last nicht zu hundert Prozent zuverlässig abläuft. Dies ist der Bluetooth-Komponente geschuldet, deren Überarbeitung in Zukunft unabdingbar ist, wenn die mobile Anwendung im Realbetrieb laufen soll. 
\\Grundsätzlich jedoch ist ein semantisches Routing in mobilen Ad-Hoc-Netzwerken möglich. Dies bestätigt die Annahme, dass ein Server für eine geordnete Kommunikation zwischen mobilen Geräten nicht zwingend benötigt wird.
\newpage
\subsection{Ausblick}
Jedes Unterkapitel im Kapitel Implementierung enthält einen Ausblick, der sich auf die jeweilige Komponente bezieht. In diesen Ausblicken wird beschrieben, inwiefern die Komponente noch erweitert und verbessert werden kann. Sie bilden zusammen einen Gesamtausblick, wobei die wichtigsten Punkte an dieser Stelle noch einmal aufgeführt werden:
\begin{itemize}
	\item Die Verbindungen über Bluetooth müssen in Zukunft stabiler werden. Dies kann unter anderem über eine Umstellung auf Bluetooth Low Energy erreicht werden, welches kein Pairing benötigt und eventuell stabiler läuft.
	\item Die durch die \textit{SQLSharkKB} bereitgestellte Persistenz sollte hinsichtlich ihrer Struktur überarbeitet werden. Der Speicherbedarf und die Abfragegeschwindigkeit könnten hierbei stark optimiert werden.
	\item Bisher können nur Textnachrichten dargestellt werden. Rohdaten können zwar Nachrichten angeheftet werden, können aber im Gesprächsverlauf noch nicht angezeigt werden.
	\item Die Benutzerfreundlichkeit im Bereich der semantischen Annotationen könnte enorm erhöht werden, wenn der Benutzer nicht mehr alles manuell eintragen muss. So könnte beispielsweise eine geeignete Vorauswahl an Themen dabei helfen, dem Benutzer per Drag and Drop Zeit zu ersparen.
	\item Nach Umsetzung der genannten Punkte ist es empfehlenswert, die Anwendung über eine Beta-Version zu testen. Die Veröffentlichung im Play-Store sollte jedoch (falls überhaupt) erst nach der Beseitigung der Bluetooth-Probleme erfolgen. 
\end{itemize}


 