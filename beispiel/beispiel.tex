\section{Dokumentengeschichte}
\begin{table}[h]
 \begin{tabular}{|l|l|l|l|}
 \hline
 Version & Autor(en) & Datum & \"Anderungen \\
 \hline
 0.0.0 & IHR NAME & DATUM & kurze Notiz  \\
 \hline
 \end{tabular}
 \caption{Dokumentengeschichte}
 \end{table}

\section{Aufgabe der Komponente}
Verbale kurze pr\"agnante Beschreibung, was die Komponente leisten soll.

(Ausf\"ullen in Prototyp-Phase)

\section{Architektur}

\subsection{\"Uberlick}
Grafik der Teile der Komponente (wichtig: Benennung aller Schnittstellen).
Anwendung der Komponente (Use Case).

\Übliche Interaktionen durch Interaktionsdiagramme.

(Ausf\"ullen in Prototyp-Phase)

\subsection{Schnittstellendefinitionen}
Beschreibung der angebotenen Schnittstellen. Benennung der Funktionen
mit Vor- und Nachbedingungen. Beschreibuung des Protocol-Bindings.

(Beginnen in Prototyp-Phase. Konkretisieren in der Alphaphase)

\subsection{genutztes Komponenten}
Beschreibung, welche Komponenten (in welchen Versionen, wo beziehbar) genutzt werden.

(Beginnen in Prototyp-Phase. Konkretisieren in der Alphaphase)

\section{Code / Deployment}
\subsection{Code}
Wo findet man den Code. Struktur des Codes. (In Prototyphase ausf\"ullen,
kann dort sehr kurz sein. Ab Alpha-Phase konkret beschreiben.)

\subsection{Deployment}
Beschreibung wie die Komponenten aus dem Quellcde erzeugt werden kann.

\section{Qualit\"atssicherung}
(Ausf\"ullen ab Alpha-Phase).

Wie erfolgt die Sicherung der Qualit\"at? Keine Romane, sondern ehrlich notieren,
was man tut. Wenn man nichts tut, dann steht hier: Wir sichern die Qualit\"at der
Komponente nicht.

Issue-Tracking: wie erfolgt das, interne Fehlermeldungen (ab Alpha),
externe Fehlermeldungen ab Beta.

\subsection{Test}
Wie wird die Komponente getestet.

\section{Vorschl\"age / Ausblick}
Was ist aufgefallen, was sollte man \"andern? L\"oschen Sie auch gern die Kommentare
der Vorg\"anger, aber nur, wenn es wirklich nicht mehr relevant ist.
