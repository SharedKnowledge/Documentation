


\subsection{Einleitung}
Durch die rasante Entwicklung des Internet of Things (IoT) ist das Interesse an einem semantischen Datenaustausch spürbar gestiegen. Wurde in den letzten Jahrzehnten noch fast ausschließlich klassisch mithilfe der Zieladresse der Datenpakete geroutet, so werden jetzt auch die Metadaten dieser Datenpakete beim Routing zunehmend beachtet. Das Routing erfolgt hierbei inhaltsbasiert und richtet sich nach den Interessen der Kommunikationsteilnehmer. Der Datenaustausch zwischen diesen Teilnehmern kann beim inhaltsbasierten Routing sowohl per klassischer Client-Server-Architektur, als auch über Peer-To-Peer (P2P) erfolgen. In dieser Arbeit wird der Datenaustausch über P2P erfolgen, was mehrere Vorteile bietet:
\begin{itemize}
\item Die Verbindungen zwischen Kommunikationsteilnehmern (Peers) können spontan aufgebaut werden. Es wird keine Serverinfrastruktur benötigt.
\item Die Daten liegen ausschließlich bei den Peers selbst. Da es keine Zwischenstation für die Datenpakete gibt, erhöht dies die Vertraulichkeit der Kommunikation immens. 
\item Nahezu alle Kommunikationsanwendungen verwenden das Internet, um den Datenaustausch zu ermöglichen. Eine Verbindung mit dem Internet ist jedoch nicht zu jeder Zeit und an jedem Ort verfügbar. Weiterhin kann auch hier auf den Zwischenservern die Kommunikation gespeichert und an Dritte weitergegeben werden.
\end{itemize}  
Der dezentrale Austausch von Daten wird unter anderem in der Android-Anwendung SharkNet realisiert. Die App verwirklicht ein dezentrales soziales Netzwerk, bei dem alle Daten ausschließlich innerhalb der Geräte gespeichert sind, es gibt keinen mithörenden zentralen Server. \\Ziel des neuen Protokolls soll es sein, dass die Benutzer der App Nachrichten an andere sich in der Nähe befindende Benutzer versenden können. Das Routing dieser Nachrichten soll inhaltsbasiert ablaufen, sodass allein die semantische Beschreibung des Nachrichteninhalts und das Interesse der Benutzer die Route vorgibt. SharkNet hat im Kommunikationsbereich dafür einige bereits lauffähige Komponenten. Für die Umsetzung des geplanten Protokolls sind jedoch neben der Anpassung von alten auch neue Komponenten erforderlich. 
\newpage
\subsection{Struktur}
Die Arbeit folgt überwiegend der klassischen Struktur von Abschlussarbeiten im Bereich der Informatik. Nach einer kurzen Einleitung werden die benötigten Grundlagen erläutert und anschließend wissenschaftliche Paper vorgestellt, welche ein ähnliches Thema haben. Das Kapitel Entwurf erklärt den Aufbau und den Geltungsbereich des Protokolls, sowie die Architektur und Oberfläche der App. Das Kapitel Implementierung beinhaltet die Komponenten, die für diese Arbeit weiterentwickelt und gänzlich neu entworfen worden sind. Die Beschreibung einer Komponente erfolgt dabei immer nach folgendem Schema:\newline
\begin{enumerate}
	\item Es wird zunächst die Aufgabe und Bedeutung der Komponente innerhalb der App dargestellt.
	\item Anschließend wird die Architektur der Komponente mit Abbildungen vorgestellt.
	\item Darauf folgen die Hinweise, inwiefern die Komponente durch andere Softwareentwickler genutzt werden kann.
	\item Für jede Komponente gibt es außerdem ein eigenes Kapitel zum Thema Test, dies umfasst je nach Komponentenart verschiedene Testarten.
	\item Abgeschlossen wird die Beschreibung der Komponente durch einen Ausblick, in dem festgehalten wird, auf welche Art und Weise die Komponente in Zukunft noch verbessert werden könnte.\\
\end{enumerate}
Im sich anschließenden Kapitel Test wird ausschließlich die gesamte Anwendung getestet, die spezifischen Tests befinden sich in den Komponentenbeschreibungen. 
\newline Abgerundet wird die Arbeit durch ein abschließendes Fazit und einen Ausblick, welcher die Chancen und Erweiterungsmöglichkeiten der Anwendung zum Inhalt hat.
\newpage


