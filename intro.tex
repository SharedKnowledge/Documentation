


\section{Einleitung}
Durch die rasante Entwicklung des Internet of Things (IoT) ist das Interesse an einen semantischen Datenaustausch spürbar gestiegen. Wurde in den letzten Jahrzehnten noch fast ausschließlich klassisch über die Zieladresse der Datenpakete geroutet, so werden jetzt auch die Metadaten dieser Datenpakete beim Routing zunehmend beachtet. Das Routing erfolgt hierbei also inhaltsbasiert und ermöglicht ein Routing nach den Interessen der Kommunikationsteilnehmer. Der Datenaustausch zwischen diesen Teilnehmern kann beim inhaltsbasierten Routing sowohl per klassischer Client-Server Architektur, als auch Peer-To-Peer (P2P) erfolgen. In dieser Arbeit wird der Datenaustausch über P2P erfolgen, was mehrere Vorteile bietet:
\begin{itemize}
\item Die Verbindungen zwischen Kommunikationsteilnehmern (Peers) können spontan aufgebaut werden, es wird keine Serverinfrastruktur benötigt.
\item Die Daten liegen ausschließlich bei den Peers selbst. Da es keine Zwischenstation für die Datenpakete gibt, erhöht dies die Vertraulichkeit der Kommunikation immens. 
\item Nahezu alle Kommunkationsanwendungen verwenden das Internet um den Datenaustausch zu ermöglichen. Eine Verbindung mit dem Internet ist jedoch nicht zu jeder Zeit und an jedem Ort verfügbar. Weiterhin kann auch hier auf den Zwischenservern die Kommunikation gespeichert und an Dritte weitergegeben werden.
\end{itemize}  


